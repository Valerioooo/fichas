\documentclass[14pt]{article}

\usepackage[a4paper, left=2cm, right=2cm]{geometry} % A4 paper size and thin margins
% assinatura

\usepackage[signature=default_signature, nojob]{mysignature}

%--------
\usepackage{xcolor} % Required for specifying custom colours
\definecolor{grey}{rgb}{0.8,0.8,1} % Colour of the box surrounding the title

\usepackage[T1]{fontenc}
\usepackage[utf8]{inputenc} % Output font encoding for international characters
\usepackage[portuguese]{babel}
\usepackage{graphicx}
\usepackage[sfdefault]{ClearSans} % Use the Clear Sans font (sans serif)
%\usepackage{XCharter} % Use the XCharter font (serif)
\usepackage{fancyhdr}
\usepackage{dirtytalk}
\usepackage{listings}
\usepackage{xcolor}
\usepackage{float}
\usepackage{hyperref}
\hypersetup{
    colorlinks=true,
    linkcolor=cyan,
    filecolor=magenta,
    urlcolor=cyan,
}
\definecolor{blue}{rgb}{0.15,0.0,1.20}
\definecolor{codegreen}{rgb}{0,0.6,0}
\definecolor{codegray}{rgb}{0.5,0.5,0.5}
\definecolor{codepurple}{rgb}{0.58,0,0.82}
\definecolor{backcolour}{rgb}{0.95,0.95,0.91}

\lstdefinestyle{mystyle}{
    backgroundcolor=\color{backcolour},
    commentstyle=\color{codegreen},
    keywordstyle=\color{magenta},
    numberstyle=\tiny\color{codegray},
    stringstyle=\color{codepurple},
    basicstyle=\ttfamily\footnotesize,
    breakatwhitespace=false,
    breaklines=true,
    captionpos=b,
    keepspaces=true,
    numbers=left,
    numbersep=5pt,
    showspaces=false,
    showstringspaces=false,
    showtabs=false,
    tabsize=2
}

\lstset{style=mystyle}

\fancyhf{}
\pagestyle{fancy}
\rfoot{\thepage\hspace{1pt}}
\begin{document}

\begin{titlepage}

	\colorbox{grey}{
		\parbox[t]{0.93\textwidth}{ % Outer full width box
			\parbox[t]{0.91\textwidth}{ % Inner box for inner right text margin
				\raggedleft
				\fontsize{60pt}{70pt}\selectfont
				\vspace{0.5cm}

				FCT: Relatório do Curso de Prática Simulada\\


				\vspace{0.5cm}
			}
		}
	}
	\vfill

	\parbox[t]{0.93\textwidth}{
		\raggedleft
		\large
		{\Large
   	Escola Secundária António Damásio\\[4pt]
    \large Curso 2: Técnicas básicas de escrita de páginas dinâmicas em PHP\\[2pt]
    \large Tecnico de Gestão e Programação de Redes Informáticas: 2º Ano\\[2pt]
    \Large Diogo Valério}\\[4pt]
    Julho de 2020\\[1pt]
		\hfill\rule{0.6\linewidth}{1pt}
	}
\end{titlepage}


\tableofcontents
\newpage

\section{Introdução} \label{introducao}
Este relatório foi realizado no âmbito do curso de prática simulada facultado pela \textbf{ANPRI} (Associação Nacional Professores de Informática), a que fomos submetidos no lugar da
Formação em Contexto de Trabalho devido à pandemia de Covid-19 a que estamos a passar de momento.

O curso frequentado foi o \textbf{Curso 2: Técnicas básicas de escrita de páginas dinâmicas em PHP} e decorreu de 19 de maio a
8 de julho, mas, devido a problemas técnicos na plataforma e também a ataques de \textit{Denial of Service}(também conhecido como \textit{DDOS})
à plataforma, o decorrer normal do curso foi prejudicado, sendo que este acabou por ser movido para a plataorma \textit{Google Sites}.


\section{Caracterização do Curso}
\subsection{Objetivos do Curso}
O curso teve como objetivo ensinar os alunos a
desenvolver páginas web dinâmicas com recurso à linguagem de programação PHP.
O curso também desenvolveu uma vertente de HTML para o desenvolvimento de páginas web e também de SQL para a criação de bases de dados.

\subsection{Area de Atividade}
O curso foi hospedado primeiramente no \textit{moodle} da \textbf{ANPRI} e seguidamente na plataforma \textit{Google Sites}.

link para o curso no \textit{moodle} : \url{https://anpri.edu.pt/course/view.php?id=58}

link para o curso no \textit{Google Sites} : \url{https://sites.google.com/anpri.pt/cursodepraticasimuladaphp/}

\subsection{Contactos}
O curso foi dirigido pelo formador \textbf{Prof. Carlos Almeida} (prof.carlos.almeida@gmail.com) e foi o desenvolvido com a ajuda da \textbf{Prof. Ester Tavares} (ester.tro@gmail.com)

\section{Recursos Materiais}
Durante o decorrer do curso foram utilizados os seguintes materiais:

\subsection{Hardware}
\begin{list}{•}
\item Computador pessoal
\end{list}

\subsection{Software}

\subsubsection{Curso}
\begin{list}{•}
\item Editor: \textit{Atom};
\item
\item Servidor: \textit{Xampp};
\item SQL: \textit{phpMyAdmin}
\item Pasta Partilhada: \textit{GitHub}.
\end{list}

\subsubsection{Manual de Utilizador}
\begin{list}{•}
  \item Editor: \textit{Atom};
  \item
  \item Linguagem: \textit{Markdown}.
\end{list}


\subsubsection{Relatório}
\begin{list}{•}
\item Editor: \textit{Atom};
\item
\item Linguagem: \textit{LaTeX}.

\end{list}

\section{Descrição das Atividades Realizadas}
\subsection{1ª Semana}
Para a realização das fichas propostas na 1ª semana foi utilizado este manual de PHP fornecido pelo formador:

(\url{https://drive.google.com/file/d/1pBc5DzuNDO3r-SwA6CAJLiIEQJIsKn2r/view})

Durante a primeira semana foram desenvolvidos programas simples pois estes tinham como propósito ajudar-nos a perceber a sintaxe da linguagem PHP.
Entre estes programas estão:
\subsubsection{Programa que devolve uma mensagem}
Este programa teve como objetivo ensinar-nos a usar a função "\textbf{echo}".
\begin{lstlisting}[language=PHP]
  <!DOCTYPE html>
  <html lang="en" dir="ltr">
    <head>
      <meta charset="utf-8">
      <title></title>
    </head>
    <body>
      <?php echo "O primeiro programa nunca se esquece!"; ?>

    </body>
  </html>

\end{lstlisting}

\subsubsection{Programa que Devolve o Resultado de uma Operação Matemática}

Este programa teve como objetivo ensinar-nos a utilizar as Operações Matemáticas, neste caso, a Adição.

\begin{lstlisting}[language=PHP]
  <!DOCTYPE html>
  <html lang="en" dir="ltr">
    <head>
      <meta charset="utf-8">
      <title></title>
    </head>
    <body>
      <?php
  $a = 23;
  $b = 45;
  echo $a+$b;
  ?>

    </body>
  </html>


\end{lstlisting}

\subsubsection{Programa que Devolve a Tabuada de um Número}

Este programa teve como objetivo ensinar-nos a utilizar as estruturas de repetição, neste caso, o ciclo "\textbf{\textit{for}}".

\begin{lstlisting}[language=PHP]
  <!DOCTYPE html>
  <html lang="en" dir="ltr">
    <head>
      <meta charset="utf-8">
      <title></title>
    </head>
    <body>
  <?php
  $a = 4;
  for ($i=0; $i <= 12; $i++) {
    echo $a, "x",$i , "=" ,$a*$i, '<br \>';
  }
  ?>

    </body>
  </html>


\end{lstlisting}

\subsubsection{Programa que Devolve o Valor Presente num Espaço de um Array}

Este programa teve como objetivo ensinar-nos a utilizar arrays.

\begin{lstlisting}[language=PHP]
  <!DOCTYPE html>
  <html lang="en" dir="ltr">
    <head>
      <meta charset="utf-8">
      <title></title>
    </head>
    <body>
  <?php

  $cores = array('Vermelho', 'Verde', 'Azul', 'Violeta');
  echo $cores[1];

  ?>

    </body>
  </html>

\end{lstlisting}

\subsubsection{Programa que Cria um formulário}

Este programa teve como objetivo ensinar-nos a utilizar formulários em conjunto com os métodos \textbf{POST} e \textbf{GET}.

\begin{lstlisting}[language=PHP]
  <!DOCTYPE html>
  <html lang="en" dir="ltr">
    <head>
      <link rel="stylesheet" href="style.css">
      <meta charset="utf-8">
      <title></title>
    </head>
    <body>
      <?php ?>
      <fieldset>
        <form action="tratamento.php" method="post">
          <table>
          <tr>
        <label for="nome">Nome: </label> <input type = "text" name="nome"><br>
        <label for="email">Email: </label> <input type = "email" name = "email"><br>
        <label for="telefone">Telefone/Telemóvel: </label> <input type = "text" name = "telefone"><br>
        <br>
        <br>
            <td align = "left">
              Cursos de informatica:
            </td align = "left">
            <td>
               <input name = "cursos" type="radio"  value="Android" > Android
               <br>
               <input name = "cursos"type="radio" value="Html" > Html
               <br>
               <input name = "cursos" type="radio"  value="Java"> Java
               <br>
               <input name = "cursos" type="radio"  value="PHP" > PHP
               <br>
               <input name = "cursos" type="radio" value="Python"> Python
               <br>
               <input name = "cursos" type="radio" value="SQL" > SQL
               <br>
               <input name = "cursos" type="radio" value="Scratch" > Scratch
               <br>
               <br>
             </td>
             <tr>
                <td>
                  <label for="nivel">Selecione uma opção:</label>
                </td>
                <td align="left">
                    <select name="nivel">
                      <option value="init">Iniciação</option>
                      <option value="med">Intermédio</option>
                      <option value="adv">Avançado</option>
                    </select>
                </td>
             </tr>

             <tr>
                <td>
                  <br>
                  <br>
                  <br>
                  <label for="obs">Observações:</label>
                  <textarea name="obs"></textarea>
                </td>
            </tr>

          </tr>
        </table>
        <input type="submit" name="submit" value="Enviar">
        </form>
      </fieldset>
    </body>
  </html>
\end{lstlisting}

\subsubsection{Programa que Define COOKIES e em Seguida Devolve-as}

Este programa teve como objetivo ensinar-nos a utilizar \textbf{COOKIES}.

\begin{lstlisting}[language=PHP]
  <!DOCTYPE html>
  <html lang="en" dir="ltr">
    <head>
      <meta charset="utf-8">
      <title></title>
    </head>
    <body>
      <?php

      $cookies['nome'] = 'joao';
      $cookies['Apelido'] = 'Alberto';
      $cookies['idade'] = '20';
      $cookies['morada'] = 'rua das 12 casas nr 13';

      foreach($cookies as $key => $value) {
        setcookie($key,$value, time()+3600);
      }
      echo "<hr /><pre>";print_r($_COOKIE);echo '<pre>';
      ?>
    </body>
  </html>

\end{lstlisting}

\subsubsection{Programa que Define Variávies de Sessão por Meio de um Formulário e em Seguida Devolve-as}

Este programa teve como objetivo ensinar-nos a utilizar \textbf{Sessões}.

\begin{lstlisting}[language=PHP]
  <!DOCTYPE html>
  <html lang="en" dir="ltr">
   <head>
     <meta charset="utf-8">
     <title></title>
   </head>
   <body>
     <form name = "form" action="" method="post">
       Nome:
       <input type = "text" name = "name">
       <input type = "submit" value= "Iniciar Sessão" name="sendform">
     </form>
     <?php
       if(isset($_SESSION['sendform'])){
         $ses['id'] = session_id();
         $ses['on'] = time();
         $ses['off'] = time() + 30;
         $ses['ip'] = $_SERVER['REMOTE_ADDR'];
         $ses['nome'] = $_POST['nome'];

         $_SESSION['user'] = $ses;

         header('Location'.$_SERVER['PHP_SELF']);
       }

       if(empty($_SESSION['user'])){
         echo '<form name = "form" action="" method="post">
               Nome:
               <input type = "text" name = "name">
               <input type = "submit" value= "Iniciar Sessão" name="sendform">
               </form>';
       }
       else{
         $tempoLog = $_SESSION['user']['on'];
         $tempoAgora = time();
         $tempoOnLine = $tempoAgora - $tempoLog;
         $tempoFim = $_SESSION['user']['off'] - $tempoAgora;
         echo 'ola'.$_SESSION['user']['nome'].' Esta logado a '.$tempoLog.'segundos';
         echo 'O seu IP é'.$_SESSION['user']['ip'];
       }
     ?>
   </body>
  </html>
\end{lstlisting}

\subsection{2ª Semana}

Para a realização das fichas propostas na 2ª semana foi utilizado este manual de PHP fornecido pelo formador:

(\url{https://drive.google.com/file/d/1KPxefR_ecFTfHb6bireXgTzu93E6CVRZ/view})

Durante a 2ª semana houve a abordagem da linguagem SQL de maneira a ensinar os alunos a criar e manipular bases de dados relacionais.
Alguns exemplos da sua utilização são:

\subsubsection{Criação de Tabelas Relacionadas }

\begin{lstlisting}[language=SQL]
  create table Colaborador(
    nomeColaborador char(30) PRIMARY KEY,
    morada char(30) not null,
    cidade char(15) not null,
    estadoCivil char(15) not null );

  create table Empresa(
    nomeEmpresa char(30) PRIMARY KEY,
    cidade char(15) not null );

    create table Trabalha(
    	nomeColaborador char(30),
    	nomeEmpresa char(30),
    	salario float not null ,
    	FOREIGN key(nomeColaborador) REFERENCES Colaborador(nomeColaborador),
      FOREIGN key(nomeEmpresa) REFERENCES Empresa(nomeEmpresa),
      PRIMARY KEY(nomeColaborador,nomeEmpresa)
    );

    create table Diretor(
      	nomeColaborador char(30),
      	nomeEmpresa char(30),
      	FOREIGN key(nomeColaborador) REFERENCES Colaborador(nomeColaborador),
        FOREIGN key(nomeEmpresa) REFERENCES Empresa(nomeEmpresa),
        PRIMARY KEY(nomeColaborador,nomeEmpresa)
    );

\end{lstlisting}

\subsubsection{Criação de Utilizadores e Definição dos seus Privilégios}

\begin{lstlisting}[language=SQL]
  CREATE USER CAntunes IDENTIFIED BY 'CAntunes?218';
  CREATE USER ASilva IDENTIFIED BY 'ASilvs!109';
  GRANT All PRIVILEGES  ON Colaborador.* TO CAntunes;
  GRANT SELECT ON Colaborador.* TO ASilva;
\end{lstlisting}
\newpage
\subsubsection{Criação de Uma Base de Dados Relacional }

\begin{lstlisting}[language=SQL]
create database Pais;


create table Distrito(
  CodDistrito int Auto_increment PRIMARY KEY,
  nomeDistrito varchar(30),
  AreaTotal float not null,
  População int not null
);

create table Provincia(
  CodProvincia int Auto_increment PRIMARY KEY,
  nomeProvincia varchar(30) not null,
  DescricaoPorvincia varchar(250)not null
);

create table Concelho(
  CodConcelho int Auto_increment PRIMARY KEY,
  CodDistrito int not null,
  nomeConcelho varchar(30) not null,
  CodProvincia int not null,
  FOREIGN key(CodDistrito) REFERENCES Distrito(CodDistrito),
  FOREIGN key(CodProvincia) REFERENCES Provincia(CodProvincia)
);

insert into Distrito(nomeDistrito, AreaTotal, Populacao) values('Lisboa', 2761, 3079772);
insert into Distrito(nomeDistrito, AreaTotal, Populacao) values('Leiria', 3506, 470895);
insert into Distrito(nomeDistrito, AreaTotal, Populacao) values('Aveiro', 2798.54, 714200);
insert into Distrito(nomeDistrito, AreaTotal, Populacao) values('Castelo Branco', 6675, 196264);
insert into Distrito(nomeDistrito, AreaTotal, Populacao) values('Coimbra', 3947, 429987);

insert into Provincia(nomeProvincia, DescricaoPorvincia)
values('Estremadura', 'A Estremadura é uma província histórica (ou região natural) de Portugal, estabelecida na Idade Média e extinta no século XIX, devendo o seu nome derivar do latim Extrema Durii');
insert into Provincia(nomeProvincia, DescricaoPorvincia)
values('Beira Litoral', 'A Beira Litoral é uma província histórica (ou região natural) situada na região do Centro de Portugal, formalmente instituída por uma reforma administrativa havida em 1936.');
insert into Provincia(nomeProvincia, DescricaoPorvincia)
values('Beira Baixa', 'A Beira Baixa é uma província histórica (ou região natural) situada na região do Centro de Portugal, originalmente criada no século XIX a partir de parte do território da anterior Província da Beira.');
insert into Provincia(nomeProvincia, DescricaoPorvincia)
values('Beira Alta', 'A Beira Alta é uma província histórica (ou região natural) situada na região do Centro de Portugal. Foi criada, em 1832, por subdivisão da antiga província da Beira, passando a ser constituída pelas comarcas de Viseu, Lamego e Trancoso.');
insert into Provincia(nomeProvincia, DescricaoPorvincia)
values('Douro Litoral', 'O Douro Litoral é uma província histórica de Portugal, formalmente instituída por uma reforma administrativa havida em 1936.');

insert into Concelho(CodDistrito, nomeConcelho, CodProvincia) values(1,'Alcobaça',1);
insert into Concelho(CodDistrito, nomeConcelho, CodProvincia) values(2,'Alcobaça',1);
insert into Concelho(CodDistrito, nomeConcelho, CodProvincia) values(3,'Castelo de Paiva',2);
insert into Concelho(CodDistrito, nomeConcelho, CodProvincia) values(4,'Fundão',3);
insert into Concelho(CodDistrito, nomeConcelho, CodProvincia) values(5,'Arganil',3);

\end{lstlisting}


\subsection{3ª Semana}
Para a realização das fichas propostas na 3ª semana foi utilizado o mesmo manual de PHP da 1ª semana:

(\url{https://drive.google.com/file/d/1KPxefR_ecFTfHb6bireXgTzu93E6CVRZ/view})

Durante a 3ª semana foi trabalhada a ligação a sockets criando clientes e servidores.
Um exemplo dos exercicios desenvolvidos é:

\subsubsection{Servidor e Cliente que }

\begin{lstlisting}[language=PHP]
  <?php
    error_reporting(E_ALL);
    set_time_limit(0);
    ob_implicit_flush();
    $address = "127.0.0.1";
    $port = "8887";
    $cliente = array();
    if( ($sock = socket_create(AF_INET, SOCK_STREAM, 0)) === false){
      echo "ERRO - Falha na criação do socket! \n".
      socket_strerror(socket_last_error())."\n";
    }
    if(socket_bind($sock, $address, $port) === false){
      echo "ERRO - Falha na passagem do Endereço e porta para o socket! \n".
      socket_strerror(socket_last_error($sock))."\n";
    }
    if(socket_listen($sock) === false){
      echo "ERRO - Falha na escuda da ligação ao socket! \n".
      socket_strerror(socket_last_error($sock))."\n";
    }
  do{
    echo "\nA esperar por clientes...\n";
    $read = array();
    $read[] = $sock;
    $write = array();
    $expect = array();
    $tv_sec = NULL;
  	$cont = 0;
  	$maxClientes = 2;
    $read = array_merge($read, $cliente);
    if(socket_select($read, $write, $expect, $tv_sec) === false){
      echo "ERRO - Falha ao aceitar a ligação ao socket! ".
      socket_strerror(socket_last_error($sock))."\n";
      continue;
    }
    if(in_array($sock, $read)){
      if(($msgsock = socket_accept($sock)) === false){
        echo "ERRO - Falha ao aceitar a ligação ao socket! ".
        socket_strerror(socket_last_error($sock))."\n";
        break;
      }
      $cliente[] = $msgsock;
      $key = array_keys($cliente, $msgsock);
  		$cont = count($cliente);
  		if ($cont > $maxClientes) {
  			echo "Nova ligacao recusada. O servidor chegou ao maximo de ", $maxClientes, " ligacoes";
  			$msgmax = "O servidor não permite mais ligações";
  			socket_write($msgsock, $msgmax, strlen($msgmax));
  			socket_close($client);
  			unset($cliente[$key]);
  		}
  		socket_getpeername($msgsock, $ip);
      echo "\n Um cliente com ip ",$ip," estabeleceu a ligação - cliente numero: $key[0]\n";
  		echo "Neste momento esta(ao) " ,$cont, " clientes conectado(s) ao servidor";
      $msg = "\n Bem-Vindo ao PHP server socket V2 - Multi - Cliente. \n\r".
      "Voce é o cliente numero: $key[0] \n\r".
      "Para sair, digite 'quit'. Para desligar o servidor digite 'shutdown'. \n\r";
      socket_write($msgsock, $msg, strlen($msg));
    }
    foreach ($cliente as $key => $client) {
      if(in_array($client, $read)){
        if(($buf = socket_read($client, 2048, PHP_NORMAL_READ)) === false){
          echo "ERRO- Falha na leitura do socket! ".
          socket_strerror(socket_last_error($client))."\n";
          break 2;
        }
        if(!$buf = trim($buf)){
          continue;
        }
        if($buf == 'quit'){
  				echo "\n O cliente $key disconectou-se!\n";
  				socket_close($client);
          unset($cliente[$key]);
          break;
        } else if($buf == 'shutdown'){
            socket_shutdown($sock);
          }
        $talkback = "Cliente $key disse: '$buf' \n";
  			foreach ($cliente as $key2) {
  				socket_write($key2,$talkback, strlen($talkback));
  			}
        echo "Cliente $key disse: '$buf' \n";
      }
    }
  }while(true);
  socket_close($sock);
  echo "o servidor encerrou";
?>
\end{lstlisting}


\subsection{4ª Semana}

Para a realização do Projeto Final da 4ª semana foi utilizado este manual de PHP fornecido pelo formador:

(\url{https://drive.google.com/file/d/1KPxefR_ecFTfHb6bireXgTzu93E6CVRZ/view})

Durante a 4ª Semana foi desenvolvido o Projeto Final do curso.
Este consistia em desenvolver uma base de dados relacional em SQL e em seguida um site para fazer a sua manipulação.
A integração do site com a base de dados foi feita na linguagem PHP.

Para a criação deste Projeto Final foi utilizado \href{https://getbootstrap.com/}{\textbf{\textit{Bootstrap 4}}} para facilitar o design do site e a implementação do mesmo:


O código fonte do Projeto está disponível neste endereço:

(\url{https://github.com/Valerioooo/fichas/tree/master/ProjetoFinal/site})


\newpage
\section{Reflexão Final}

No inicio do curso as atividades foram bem definidos como mostra a figura seguinte:

\begin{figure}[H]
    \centering
    \includegraphics[width=0.45\linewidth]{cal.png}
    \caption{Calendário da primeira semana}
    \label{fig:cal}
\end{figure}


Mesmo assim estes não foram cumpridos por parte do formador:

\begin{itemize}
  \item No primeiro dia (3ª feira) apenas existia forma de marcar presença para o segundo (4ª feira) e no dia seguinte foi adicionada a marcação de presença para o dia anterior;
  \item Não existiam atividades propostas nos primeiros dias;
  \item Não existiu qualquer interação entre o formador e os alunos durante as primeiras duas semanas;
  \item Existiu então uma reunião na plataforma \textit{Zoom}, à qual nem eu nem muitos alunos fomos convidados/notificados, que acabou por não ser efetuada pois o limite da plataforma é de 100 participantes enquanto o numero de alunos do curso acabava por exeder os 200;
  \item O formador não respondeu a nenhuma dúvida no forum de dúvidas até perto da 3ª semana do curso;
  \item Nunca exitiu qualquer tipo de teste ou "miniteste" durante o decorrer do curso.
\end{itemize}
\newline
Também existiram vários problemas referentes à organização do curso:
\begin{itemize}
  \item As fichas fornecidas aos alunos não eram claras em certos aspetos;
  \item Não existiu qualquer resposta da ANPRI em relação aos problemas enfrentados na utilização do site devido aos ataques sucessivos descritos no ponto \large\ref{introducao}.
\end{itemize}
\newpage
Durante o curso encontrei poucas dificuldades associadas com as fichas, exceto na ficha em que eram tratados os \textit{Sockets}:
O servidor Xampp não habilita os \textit{Sockets} por defeito. Sendo necessário que estes sejam habilitados por meio da edição do ficheiro php.ini se o Sistema Operativo utilizado for \textit{Windows}, e a compilação do código fonte da linguagem PHP com a \textit{flag} \say{--enable-sockets} caso o Sistema Operativo utilizado for MacOS ou Linux.

A Prof. Ester Tavares ajudou-me bastante na execução desta ficha e também no desenvolvimento do Projeto Final de curso.

O Projeto Final foi a parte mais interessante do curso na minha opinião, pois este proporcionou-me a descobrir a ferramenta {\textbf{\textit{Bootstrap}} (recomendada pela Prof. Ester Tavares) que se revelou uma grande ajuda ao desenvolvimento do Site.

Deixando de lado os problemas existentes, o curso revelou-se bastante interessante em termos de aprendizagem autónoma pelo facto da maioria das tarefas terem sido executadas apenas com recurso à documentação especifíca das ferramentas, a Tutoriais em vídeo e a fóruns de programação na Internet.

Pelo facto de ter entregue todos os trabalhos completos no devido prazo e por ter consciência de que executei as tarefas com o maior zelo pela qualidade das mesmas, penso que mereço uma nota de fim de estágio de 18 valores.
\newline
\begin{figure}[H]
    \centering
    \includegraphics[width=0.80\linewidth]{certificado.png}
    \caption{Certificado do Curso}
    \label{fig:certificado}
\end{figure}


\mysignature[left]{full}


\end{document}
